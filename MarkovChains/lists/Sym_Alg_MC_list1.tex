\documentclass[a4paper,12pt]{article}
\usepackage[utf8]{inputenc}
\usepackage[T1]{fontenc}
\usepackage{amssymb}
\usepackage{graphicx}
\usepackage{pstricks}
 \usepackage{amsthm}
 
 % macro, packages
\input{setup/packages3}
\input{setup/macros3}
%\usepackage[nohead]{geometry}
%\usepackage{amsmath,amsthm,amssymb,euler}

% \newtheorem{deff}{Definicja}[subsection]
% \newtheorem{twr}[deff]{Twierdzenie}
% \newtheorem{lem}{Lemat}
% \newtheorem{uwaga}[deff]{Uwaga}
% \newtheorem{alg}[deff]{Algorytm}


\newtheorem{deff}{Definition}[subsection]
\newtheorem{twr}[deff]{Theorem}
\newtheorem{lem}{Lemma}
\newtheorem{uwaga}[deff]{Remark}
\newtheorem{alg}[deff]{Algorithm}

%\newcommand{\e}[0]{\mathbf{e}}
\newcommand{\s}[0]{\mathbf{s}}
%\newcommand{\PP}[0]{\mathbf{P}}
%\newcommand{\E}[0]{\mathbb{E}}
 
\renewcommand\theequation{\thesection.\arabic{equation}}
%\setlength{\textwidth}{16cm}
%\setlength{\oddsidemargin}{3cm}
%\setlength{\evensidemargin}{3cm}
%\setlength{\hoffset}{-1in} %


\addtolength{\voffset}{-2.0cm}
\addtolength{\hoffset}{-1.0cm}
\addtolength{\textwidth}{2.0cm}
\addtolength{\textheight}{3.0cm}



\setlength{\fboxrule}{11cm}

\usepackage{fancyhdr}
\pagestyle{fancy}
\renewcommand\headrulewidth{0pt}
\fancyhead[LE,LO,RE,RO]{}
\fancyfoot[LE,LO]{\tiny {\tt\jobname    }}
\fancyfoot[RE,RO]{\tiny {\tt \leftmark     }}
%\rightmark - \leftmark --
\fancyfoot[c]{\thepage   }

\newcount\cnt
\def\nn {{
{\message{-\the\cnt-}}\par
\bf \the\cnt .\   \global\advance\cnt by 1}}
\def\nx {{
{\message{-\the\nu-}}\hskip 10truept
\bf \the\cnt .\   \global\advance\cnt by 1}}
\def\ny {{
{\message{-\the\cnt-}}\par\noindent
\bf \the\cnt .\   \global\advance\cnt by 1}}
\def\nz {{\hskip -2truept
{\message{-\the\cnt-}}
\bf \the\cnt .\   \global\advance\cnt by 1}}

\def\bp {\bigskip\par}

\usepackage{setspace}

\parindent 0pt

\everymath{\displaystyle}


\begin{document}
 
\noindent
 {
\setlength\fboxsep{4pt}%
 \setlength\fboxrule{2pt}%
 \fbox{%
  \begin{tabular}{lcr}
\multicolumn{2}{l}{\bf  Simulations and algorithmic applications of Markov chains} &    2025/26   \\ \\
\multicolumn{3}{c}{   List nr 1} \\
\multicolumn{2}{l}{ Paweł Lorek} &  \\  
 \end{tabular}%
}} \bigskip\bigskip
\par \bigskip

%\setstretch{1}
 
 
\ \ \ \textbf{Probability}

\begin{enumerate}

\item Let  $T$ be a random variable taking non-negative integer values.
Show that  $ET=\sum_{k=0}^\infty P(T>k)$.


\item Prove the following
\begin{lem}[Markov's inequality]  Let  $X$ be a non-negative random variable. Then
 $\forall r>0$
$$P(X\geq r)\leq{EX\over r}$$
\end{lem}


\item   Prove the following
\begin{lem}[Chebyshev's inequality] Let $X$ be a random variable with mean $\mu<\infty$
and variance $\sigma^2<\infty$.
Then  $\forall r>0$
$$P(|X-\mu|\geq r)\leq{\sigma^2\over r^2}$$
\end{lem}

\item Show that 
\begin{itemize}
 \item for each $x\in\mathbb{R}$ we have  $1+x\leq e^x$,
 \item for each $x\geq 0$ we have  $\ln(1+x)\geq {x\over 1+x/2}$.
\end{itemize}




\item  Prove the following
\begin{lem}[Chernoff  inequality ] Let $X_i$ be iid random 
varible with distribution $P(X_i=1)=p_i=1-P(X_i=0)$.
Let $S=\sum_{i=1}^n X_i$ and  $\mu=ES=\sum_{i=1}^n EX_i=\sum_{i=1}^n p_i$. 
Then, for each  $\varepsilon>0$ we have 
$$P(S>(1+\varepsilon)\mu)\leq\left(e^\varepsilon\over (1+\varepsilon)^{1+\varepsilon}\right)^\mu$$ 
and
$$P(S<(1-\varepsilon)\mu)\leq\left(e^{-\varepsilon}\over (1-\varepsilon)^{1-\varepsilon}\right)^\mu$$ 

 
\end{lem}

\item Show that previous exercise implies that for $\varepsilon\in(0,1)$ we have
$$P(S<(1-\varepsilon)\mu)\leq e^{-{\varepsilon^2\over 2}\mu}$$
and
$$P(S<(1+\varepsilon)\mu)\leq e^{-{\varepsilon^2\over 2+\varepsilon}\mu}$$

\item Using previous exercises show the following

\begin{lem}   Let $Y_1,\ldots,Y_R$ be i.i.d indicator variables with $p=\E Y_i$ and $\hat{Y}_R={1\over R}\sum_{i=1}^R Y_i.$
Fix   $\varepsilon, \delta\in(0,1)$. If  $R\geq{3\ln{2\over \delta}\over \varepsilon^2 p}$,
then
$$\Prob\left(|\hat{Y}_R-p|\geq \varepsilon p\right)\leq \delta.$$
\end{lem}





\bigskip\par 
\textbf{Markov chains}

 
\item\label{zad_wyk} Consider a simple random walk on 
$\E=\{1,2,3,4\}$ with transition matrix 
$$\PP=\left[
\begin{array}{cccc}
     0 & 1/2 & 0 & 1/2  \\[5pt]
      1/2 & 0 & 1/2 & 0  \\[5pt]
     0 & 1/2 & 0 & 1/2  \\[5pt]
      1/2 & 0 & 1/2 & 0  \\
    \end{array}\right]
$$ and initial distribution $\mu=(1,0,0,0)$

\begin{itemize}
 \item[a)] Compute $\mu^3$
 \item[b)] Compute $\mu^n$ (``geussing'' the solution and proving it is correct by induction).
\end{itemize}

 \item\label{zad_gothenburg}[Gothenburg model of weather]. 
 Consider a chain on two states $\E=\{\textrm{rainy},\textrm{sunny}\}$.
 After a rainy day there is again rainy day with probability 3/4, whereas after a sunny day 
 there is a sunny day with alsow a probability 3/4. 
 Let us enumerate the states $\e_1=\textrm{rainy}, \e_2=\textrm{sunny}$. Then, the transition matrix is
$$\PP=\left[
\begin{array}{cc}
     {3/4} & {1/ 4} \\[5pt]
{1/ 4} & {3/ 4}
    \end{array}\right]
$$
Assume  $\mu=(0,1)$.
\begin{itemize}
 \item[a)] Compute $\mu^2$ for initial distribution  $\mu=(0,1)$.
 \item[b)] Compute $\mu^n$ for initial distribution  $\mu=(0,1)$ using diagonalization of $\PP$.
 \item[c)] Using b), compute $\lim_{n\to\infty}\mu^n$ for any initial distribution.
\end{itemize}
 
\item\label{zad_LA}[LA (Los Angeles) model of weather]. As in previous exercise 
$\E=\{\e_1,\e_2\}=\{\textrm{rainy},\textrm{sunny}\}$, the transition matrix is
$$\PP=\left[
\begin{array}{cc}
     {1/2} & {1/ 2} \\[5pt]
{1/ 10} & {9/ 10}
    \end{array}\right]
$$
Provide a formula for $\mu^k$ for initial distribuions:
\par 
a) $\mu=(1,0), \qquad\qquad\qquad$ b) $\mu=(1/2,1/2), \qquad\qquad \qquad$ c) $\mu=(1/6, 5/6)$
\medskip\par 
Anything interesting for one of the cases?
\smallskip\par 
What does $\mu^k$ converge to in each case, as $k\to\infty$? 
 

\item
Let $(X_0, X_1,\ldots)$ be a time-homogenous Markov chain 
with transition matrix $\PP$ and state space
$\E=\{\e_1,\ldots,\e_M\}$, i.e., $P(X_{n+1}=\e_j | X_n=\e_i)=\PP(\e_i,\e_j)$.
Show that   $$P(X_{n+m}=\e_j | X_n=\e_i)=\PP^m(\e_i,\e_j)$$


\item For card shuffling scheme Top-To-Random  
(one step: take top card and put it in a deck \textsl{uniformly at random}) for $n=3$ cards 
provide a transition matrix $\PP$ and draw a transition graph. Compute also $\PP^2$.

\item For two probabilistic measures  $\mu,\nu$ on $\E=\{\e_1,\ldots,\e_M\}$ 
we define a  \textsl{total variation distance }: $d_{TV}(\mu,\nu)={1\over 2}\sum_{\e\in E} |\mu(\e)-\nu(\e)|$. \par 
Compute total variation distance between distributions:
$\mu\sim Bin(4,1/2), \nu\sim Bin(4,1/4)$

\item On a state space  $E=\{1,2\}$ we define: $\mu^k=\left[{1\over 6}+{1\over 3}\left({2\over 5}\right)^k,{5\over 6}-{1\over 3}\left({2\over 5}\right)^k\right]$ and  $\nu=\left[{1\over 6}, {5\over 6}\right]$.
\begin{itemize}
 \item[a)] Compute z $d_{TV}(\mu^k,\nu)$.
 \item[b)] For what (integer) $k$ the total variation distance between  $\mu^k$ and $\nu$ 
 is smaller than 0.01?
\end{itemize}

\item Consider $n$ cards. Let  $\mu$ be a uniform distribution 
on all $n$ permutations of the cards. Let $\mu$ be a uniform distribution 
on all permutations such that first card is fixed.\par 
Compute  $d_{TV}(\mu,\nu)$.
\end{enumerate}
 




\end{document}
