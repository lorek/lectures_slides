\documentclass[a4paper,12pt]{article}
\usepackage[utf8]{inputenc}
\usepackage{enumitem}
\usepackage[T1]{fontenc}
\usepackage{amssymb}
\usepackage{graphicx}
\usepackage{pstricks}
 \usepackage{amsthm}
 
%\usepackage[nohead]{geometry}
%\usepackage{amsmath,amsthm,amssymb,euler}

% \newtheorem{deff}{Definicja}[subsection]
% \newtheorem{twr}[deff]{Twierdzenie}
% \newtheorem{lem}{Lemat}
% \newtheorem{uwaga}[deff]{Uwaga}
% \newtheorem{alg}[deff]{Algorytm}
 \usepackage{algorithm}
\usepackage{algorithmic}
%Change: REQUIRE -> Input and ENSURE -> Output
%\renewcommand{\algorithmicrequire}{\textbf{Input:}}
%\renewcommand{\algorithmicensure}{\textbf{Output:}}


\newtheorem{deff}{Definition}
\newtheorem{theorem}{Theorem}
\newtheorem{lem}{Lemma}
\newtheorem{uwaga}[deff]{Remark}
\newtheorem{alg}[deff]{Algorithm}

%\newcommand{\e}[0]{\mathbf{e}}
%\newcommand{\s}[0]{\mathbf{s}}
%\newcommand{\PP}[0]{\mathbf{P}}
%\newcommand{\E}[0]{\mathbb{E}}
 
\renewcommand\theequation{\thesection.\arabic{equation}}
%\setlength{\textwidth}{16cm}
%\setlength{\oddsidemargin}{3cm}
%\setlength{\evensidemargin}{3cm}
%\setlength{\hoffset}{-1in} %


\addtolength{\voffset}{-2.2cm}
\addtolength{\hoffset}{-1.0cm}
\addtolength{\textwidth}{2.0cm}
\addtolength{\textheight}{3.2cm}



\setlength{\fboxrule}{11cm}

\usepackage{fancyhdr}
\pagestyle{fancy}
\renewcommand\headrulewidth{0pt}
\fancyhead[LE,LO,RE,RO]{}
\fancyfoot[LE,LO]{\tiny {\tt\jobname    }}
\fancyfoot[RE,RO]{\tiny {\tt \leftmark     }}
%\rightmark - \leftmark --
\fancyfoot[c]{\thepage   }

\newcount\cnt
\def\nn {{
{\message{-\the\cnt-}}\par
\bf \the\cnt .\   \global\advance\cnt by 1}}
\def\nx {{
{\message{-\the\nu-}}\hskip 10truept
\bf \the\cnt .\   \global\advance\cnt by 1}}
\def\ny {{
{\message{-\the\cnt-}}\par\noindent
\bf \the\cnt .\   \global\advance\cnt by 1}}
\def\nz {{\hskip -2truept
{\message{-\the\cnt-}}
\bf \the\cnt .\   \global\advance\cnt by 1}}

\def\bp {\bigskip\par}

\usepackage{setspace}

\parindent 0pt

\everymath{\displaystyle}

\input{setup/packages2_noalg}
\input{setup/macros2}
 
\begin{document}
 
\noindent
 {
\setlength\fboxsep{4pt}%
 \setlength\fboxrule{2pt}%
 \fbox{%
  \begin{tabular}{lcr}
\multicolumn{2}{l}{\bf  Simulations and algorithmic applications of Markov chains} &    2024/25  \\ \\
\multicolumn{3}{c}{   List nr 10 } \\
\multicolumn{2}{l}{ Paweł Lorek} &  \\  
 \end{tabular}%
}}
\bigskip\bigskip
 
% \textbf{Ising model}. Let $G=(V,K)$ be a graph. Ising model provides a way 
% of choosing a random element $\e$ from $\{-1,+1\}^V$, called a \textsl{configuration}.
% For a given configuration $\e$ by $e(v)\in\{-1,+1\}$ we denote its (spin) value at vertex $v$.
% Fix some $\beta\neq 0$. For a given configuration we define its energy by
% $$H(\e)=-\sum_{(w,w')\in K} e(w)e(w').$$
% The Ising model is a distribution 
% $$\pi(\e)={1\over C} \exp\left(-\beta H(\e)\right)
% ={1\over C} \exp\left(\beta \sum_{(w,w')\in K} e(w)e(w')\right).$$
% 

\par \bigskip 
We consider a Markov chain $X$ on a finite state space $\E=\{\e_1,\ldots,\e_M\}$.
We assuma that $\E$ is partially ordered by $\preceq$. Moreover,
we assume that 
$$\forall(\e\in\E)\qquad \e_1\preceq \e\preceq \e_M.$$
Given $X_k$ we simulate $X_{k+1}$ using $U_{k+1}\sim\mathcal{U}(0,1)$ and an \textsl{update function} $\psi: \E\times[0,1]\to\E$ in the following way:
$$X_{k+1}=\psi(X_k,U_{k+1}).$$
Recall, $\psi$ is an update function for a Markov chain $X$ with t.m. $\mathbf{P}$ 
if 
\begin{enumerate}
 \item for fixed $\e_i$, the function $\psi(\e,u)$ is piecewise constant 
 (as a function of $u$), and
 \item For all $\e_i, \e_j\in\E$ we have $\displaystyle \int_0^1 \boldsymbol{1}(\psi(\e_i,u)=\e_j)\; du = \mathbf{P}(\e_i,\e_j).$
\end{enumerate}
Three more definitions:
\medskip\par 
\begin{deff}
 An update function $\psi$ is \textbf{monotone} w.r.t $\preceq$ if
 $$\forall(\e_i\preceq\e_j\in E) \forall(u\in[0,1])\qquad \psi(\e_i,u)\preceq\psi(\e_j,u).$$
\end{deff}
\medskip\par 
\begin{deff}
A Markov chain with a t.m. $\mathbf{P}$ is \textbf{realizable monotone} w.r.t. $\preceq$ if there exists a corresponding monotone (w.r.t. $\preceq$) update 
function $\psi$.
\end{deff}
\medskip\par 
Recall also a coupling from the past (CFTP) algorithm.

\medskip\par 
\begin{algorithm}[H]
\caption{Efficient coupling from the past (CFTP).} \label{alg:cftp_eff}
\begin{algorithmic}[1] 
\REQUIRE State space $\E$, ergodic chain $X$, a monotone update function $\psi$
\STATE Set $n=1$
\STATE Start two chains at time  $-N_n$, one at the minimum $\e_1$, 
the other at the maximum $\e_M$.
Run the chains  till time 0 using the same update 
rule $\phi$ and iid random variables $U_{-N_n+1}, U_{-N_n+2},\ldots, U_{-1}, U_0$ 
uniformly distributed on $[0,1]$ (the same for each chain).
\STATE If both chains in previous step end up in the same state $\e$ at time $0$, then output $\e$ and \textbf{stop}. 
\STATE Set $n=n+1$ and go to Step 2  (keep previously used $\{ U_i \}_{\leq -N_n+1\leq i \leq 0}$
for new $n$).
\end{algorithmic}
\end{algorithm}

\newpage 
Two more definitions
\begin{deff}
A set $A\subseteq \E$ is an \textbf{upset} if 
 $$(\e \preceq \e', \e\in A) \ \Rightarrow \ \e'\in A.$$
\end{deff}

\begin{deff}
A Markov chain with a t.m. $\mathbf{P}$ is \textbf{stochastically monotone} w.r.t. $\preceq$ if
 $$\forall (\e_i\preceq \e_j) \forall (A - \textrm{upset})\quad \mathbf{P}(\e_i,A)\leq \mathbf{P}(\e_j,A).$$
\end{deff}

\medskip\par 
\begin{enumerate}
\item

 Recall the definition of the birth and death process on a state space $\E=\{0,1,\ldots,M\}$ with the transition probability matrix
$$\PP=\left[
\begin{array}{ccccccccc}
1-p_0 & p_0 & 0 & 0 & 0  & 0 &\ldots & 0 \\[5pt]
q_1 & r_1 & p_1 & 0 & 0 & 0 &\ldots & 0 \\[5pt]
0 & q_2 & r_2 & p_2 & 0 & 0 &\ldots & 0\\[5pt]
  &      & \ddots& \ddots& \ddots\\[5pt]
0 & 0 & 0 & q_i & r_i & p_i &0 \ldots &0 \\[5pt]
 &  & &      & \ddots& \ddots& \ddots\\[5pt]
 0&\ldots &  & &      & q_{N-1} & r_{N-1} & p_{N-1} \\[5pt]
 0&\ldots &  & &      & 0 & q_0 & 1-q_0 \\[5pt]
    \end{array}\right],$$
    where $p_0>0, q_0>0$ and for all $i\in\{1,\ldots,M-1\}$: $p_i>0, q_i>0, p_i+q_i+r_i=1.$ 
    
\begin{enumerate}
 \item When (what is a restriction on $p_i,q_i$) is a birth and death chain   stochastically monotone w.r.t. linear 
 ordering $\preceq:=\leq$?
 
 \item Provide some update function for birth and death chain. Check if it is monotone.
 If it is, describe shortly a CFTP algorithm.
 
 \item Show that realizable monotonicity implies stochastic monotonicity.
 
 \item (*) Show that for linear ordering $\preceq:=\leq$ stochastic monotonicity 
 of a chain is equivalent to realizable monotonicity.
 
 
     
\end{enumerate}


\item\label{exer:non-sym_square_real_mon} (Non-symmetric random walk on a square).
Consider a state space  $\mathcal{S}=\{0,1\}^2 = \{0 \equiv (0,0), 1 \equiv (1,0), 2 \equiv (0,1), 3 \equiv (1,1)\}$.
Let $\alpha\in(0,1/2), \beta\in(0,1/2)$. Define a non-symmetric random walk $(X_k)$ on $\mathcal{S}$ with a transition matrix:
$$
\mathbf{P}=\left(
\begin{array}{ccccc}
1-2\alpha & \alpha & \alpha & 0 \\
\beta& 1-\alpha-\beta & 0  & \alpha \\
\beta& 0& 1-\alpha-\beta & \alpha \\
0&\beta& \beta& 1-2\beta & \\
\end{array}\right)
$$
Roughly speaking, being in $s=(e_1,e_2)$, where $e_i\in\{0,1\}$, the chain may change a coordinate with value
0 to 1 with probability $\alpha$ and 1 to 0 with probability $\beta$, or it does nothing with the remaining probability.

Let us introduce coordinate-wise ordering. For $\e = (e_1, e_2)$ and $\e' = (e'_1, e'_2)$, define:
$$\e \preceq \e' \ \Leftrightarrow e_1' \leq e_1' \textrm{ and } e_2' \leq e_2'.$$
Note that $0\equiv (0,0)$ is the minimum and $3\equiv (1,1)$ is the maximum.

\medskip\par\noindent
Show that $(X_k)$ is realizable monotone if $\alpha+\beta\leq 1/2$, provide explicitly a monotone update function $\varphi$.

\item Reconsider a random walk $(X_k)$ from Exercise \ref{exer:non-sym_square_real_mon}.
For what values of $\alpha$ and $\beta$ is the chain stochastically monotone?

\end{enumerate}





\end{document}
