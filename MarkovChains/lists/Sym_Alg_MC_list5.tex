\documentclass[a4paper,12pt]{article}
\usepackage[utf8]{inputenc}
\usepackage{enumitem}
\usepackage[T1]{fontenc}
\usepackage{amssymb}
\usepackage{graphicx}
\usepackage{pstricks}
 \usepackage{amsthm}
 
%\usepackage[nohead]{geometry}
%\usepackage{amsmath,amsthm,amssymb,euler}

% \newtheorem{deff}{Definicja}[subsection]
% \newtheorem{twr}[deff]{Twierdzenie}
% \newtheorem{lem}{Lemat}
% \newtheorem{uwaga}[deff]{Uwaga}
% \newtheorem{alg}[deff]{Algorytm}


\newtheorem{deff}{Definition}[subsection]
\newtheorem{theorem}{Theorem}
\newtheorem{lem}{Lemma}
\newtheorem{uwaga}[deff]{Remark}
\newtheorem{alg}[deff]{Algorithm}

%\newcommand{\e}[0]{\mathbf{e}}
%\newcommand{\s}[0]{\mathbf{s}}
%\newcommand{\PP}[0]{\mathbf{P}}
%\newcommand{\E}[0]{\mathbb{E}}
 
\renewcommand\theequation{\thesection.\arabic{equation}}
%\setlength{\textwidth}{16cm}
%\setlength{\oddsidemargin}{3cm}
%\setlength{\evensidemargin}{3cm}
%\setlength{\hoffset}{-1in} %


\addtolength{\voffset}{-2.0cm}
\addtolength{\hoffset}{-1.0cm}
\addtolength{\textwidth}{2.0cm}
\addtolength{\textheight}{3.0cm}



\setlength{\fboxrule}{11cm}

\usepackage{fancyhdr}
\pagestyle{fancy}
\renewcommand\headrulewidth{0pt}
\fancyhead[LE,LO,RE,RO]{}
\fancyfoot[LE,LO]{\tiny {\tt\jobname    }}
\fancyfoot[RE,RO]{\tiny {\tt \leftmark     }}
%\rightmark - \leftmark --
\fancyfoot[c]{\thepage   }

\newcount\cnt
\def\nn {{
{\message{-\the\cnt-}}\par
\bf \the\cnt .\   \global\advance\cnt by 1}}
\def\nx {{
{\message{-\the\nu-}}\hskip 10truept
\bf \the\cnt .\   \global\advance\cnt by 1}}
\def\ny {{
{\message{-\the\cnt-}}\par\noindent
\bf \the\cnt .\   \global\advance\cnt by 1}}
\def\nz {{\hskip -2truept
{\message{-\the\cnt-}}
\bf \the\cnt .\   \global\advance\cnt by 1}}

\def\bp {\bigskip\par}

\usepackage{setspace}

\parindent 0pt

\everymath{\displaystyle}

\input{setup/packages2}
\input{setup/macros2}
\begin{document}
 
\noindent
 {
\setlength\fboxsep{4pt}%
 \setlength\fboxrule{2pt}%
 \fbox{%
  \begin{tabular}{lcr}
\multicolumn{2}{l}{\bf  Simulations and algorithmic applications of Markov chains} &     2024/25  \\ \\
\multicolumn{3}{c}{   List nr 5} \\
\multicolumn{2}{l}{ Paweł Lorek} &  \\  
 \end{tabular}%
}} \bigskip\bigskip
\par \bigskip

%\setstretch{1}
 
  For a transition matrix $\PP$  of size $M\times M $ we denote its eigenvalues in non-increasing (w.r.t. absolute value) 
  order: $|\lambda_1|\geq|\lambda_2|\geq \ldots\geq |\lambda_M$.
\medskip\par 
  \textbf{Poincare constant}. Let $X$ be a reversible Markov chain with a transition matrix $\PP$ on a graph $G=(V,K)$, where $V=\{\e_1,\ldots,\e_M\}$ are the vertices and $K=\{(\e_i,\e_j): \PP(\e_i,\e_j)>0\}$ are the edges. For a directed edge $\tilde{k}=(\e_i,\e_j)$ ($\e_i$ is the starting edge) define $\Lambda(\tilde{k})=\pi(\e_i)\PP(\e_i,\e_j)$. For set verices $\e_i$ and $\e_j$ let $\Gamma(\e_i,\e_j)$ be a deterministic, unique path. Let $|\Gamma(\e_i,\e_j)|=\sum_{\tilde{k}\in\Gamma(\e_i,\e_j)} 1$ be a length of that path.
Poincare constant is defined as 
$$\gamma_P:=\max_{\tilde{k}} \left\{ {1\over \Lambda{(\tilde{k})}} \sum_{(\e_i,\e_j): \tilde{k}\in\Gamma(\e_i,\e_j)} |\Gamma(\e_i,\e_j)| \pi(\e_i)\pi(\e_j) \right\}$$
(for set $\tilde{k}$ the sum goes through all the vertices $(\e_i,\e_j)$ such that the edge $\tilde{k}$ belongs to the path $\Gamma(\e_i,\e_j)$).

\begin{theorem}
Let $\lambda_2$ denote the second largest (in terms of absolute value) eigenvalue of a transition matrix $\PP$ of a reversible Markov chain $X$. Then
$$|\lambda_2|\leq 1-{1\over\gamma_P}.$$
\end{theorem}

Similarly to the Poincare constant, for a reversible and ergodic Markov chain $X\sim \PP$  one can define  \par 
\textbf{Cheeger constant}
$$\gamma_C:=\min_{A\subset\E: \pi(A)\leq 1/2} { \displaystyle  \sum_{\e_i\in A} \sum_{\e_j\in A^C} \pi(\e_i)\PP(\e_i,\e_j) \over \pi(A)}.$$
\begin{theorem} With $\lambda_2$ defined as previously we have
$$|\lambda_2|\leq 1-{1\over 2}\gamma_C^2.$$
\end{theorem}
 
 Having a bound on $|\lambda_2|$ we have a bound on total variation distance:
\begin{theorem} For a reversible Markov chain $X\sim\PP$ we have
$$d_{TV}(\delta_\e\PP^n,\pi)\leq {1\over 2} {1\over \sqrt{\pi(\e)}} |\lambda_2|^n,$$
where $\delta_\e\PP^n$ is the distribution of the $n-$th step of the chain that started in state $\e$.
\end{theorem}

% \begin{theorem} For any (not necessarily reversible!) Markov chain $X\sim\PP$ we have 
% $$d_{TV}(\delta_\e\PP^n,\pi)\leq {1\over 2} {1\over \sqrt{\pi(\e)}} |\lambda_2(M)|^{n\over 2},$$
% where $\lambda_2(M)$ is the second largest (in terms of absolute value) eigenvalue of a matrix $M=\PP\tilde{\PP}.$
% \end{theorem}
\newpage 



\begin{enumerate}
\item 
  For $\phi:\E\to\mathcal{R}$ we have defined 
  $Var(\phi)={1\over 2}\sum_{\e_i, \e_j\in\E} (\phi(\e_i)-\phi(\e_j))^2 \pi(\e_i)\pi(\e_j).$\smallskip\par 
  Show that 
  $Var(\phi)=\sum_\e \phi^2(\e)\pi(\e)-\left(\sum_\e\phi(\e)\pi(\e)\right)^2.$ 

  
  \item
  Show that for $n_{\gamma_P}(\varepsilon)=\gamma_P \log\left({1\over 2\varepsilon\sqrt{\pi(\e)}}\right)$
$$d_{TV}(\delta_\e\PP^{n_{\gamma_P}(\varepsilon)},\pi)\leq \varepsilon,$$
where $\gamma_P$ is the Poincare constant.

\item
Show that for $n_{\gamma_C}(\varepsilon)={2\over \gamma_C^2} \log\left({1\over 2\varepsilon\sqrt{\pi(\e)}}\right)$
$$d_{TV}(\delta_\e\PP^{n_{\gamma_C}(\varepsilon)},\pi)\leq \varepsilon,$$
where $\gamma_C$ is the Cheeger constant.
 
\item  Consider the symmetric random walk on a circle: $\E=\{0,1,\ldots,n-1\}$.
$$\PP=\left[
\begin{array}{ccccccccc}
1-2p & p & 0 & 0  & \ldots & 0& 0 & p\\[7pt]
p & 1-2p & p & 0 & \ldots & 0 &0&0\\[7pt]
0 & p & 1-2p & p & 0 &\ldots & 0 &0\\[7pt]
  &    &     &   &   & \ddots & & \\[7pt]
p & 0 & 0 & 0 & 0 &\ldots & p &1-2p\\[7pt]
\end{array}\right],$$
where $p<\frac{1}{2}.$ Calculate (or estimate) $\gamma_P.$
 
\item For previous exercise, recall (from lecture) the formula for  $\gamma_C$. 
For what $p$ the Cheeger constant is better that the Poincare constant in the case of random walk on a circle (by better we mean that it gives better estimation in Theorem 2).
%  
% \item Once again consider a random walk on a circle: $\E=\{0,1,\ldots,n-1\}$
% $$\PP=\left[
% \begin{array}{ccccccccc}
% 1-p & p & 0 & 0  & \ldots & 0& 0 & 0\\[7pt]
% 0 & 1-p & p & 0 & \ldots & 0 &0&0\\[7pt]
% 0 & p & 1-p & p & 0 &\ldots & 0 &0\\[7pt]
%   &    &     &   &   & \ddots & & \\[7pt]
% p & 0 & 0 & 0 & 0 &\ldots & p &1-p\\[7pt]
% \end{array}\right].$$
% Show that by calculating $\mathbf{M}=\PP\tilde{\PP}$ the task to calculate $\gamma_C$ and $\gamma_P$ reduces to exercises 4 and 5. Give the estimations.
 
\item Let $X$ be a simple random walk on a graph $G=(V,K),$ where $V=\{\e_1,\ldots,\e_M\}$ are the vertices and $K=\{(\e_i,\e_j): \PP(\e_i,\e_j)>0\}$ are the edges. Let the transition matrix be $\PP(v_i,v_j)=1/d(v_i)$ if $(v_i,v_j)\in K,$ where $d(v_i)$ is the degree of vertex $v_i.$ Let $\Gamma(v,v')$ denote some choice of a path from $v$ to $v'$ which doesn't include the same edge more than once. Define 
$$d^*=\max_v d(v),\qquad  s^*=\max_{v,v'} |\Gamma(v,v')|,  \qquad \eta^*=\max_{\tilde{k}\in K} \#\{(v,v')\in V^2: \tilde{k}\in\Gamma(v,v')\}.$$
Show that $\gamma_P\leq {(d^*)^2 s^* \eta^*\over |K|}.$
 
 \end{enumerate}




\end{document}
