\documentclass[a4paper,12pt]{article}
\usepackage[utf8]{inputenc}
\usepackage[T1]{fontenc}
\usepackage{amssymb}
\usepackage{graphicx}
\usepackage{pstricks}
 \usepackage{amsthm}
 
%\usepackage[nohead]{geometry}
%\usepackage{amsmath,amsthm,amssymb,euler}

% \newtheorem{deff}{Definicja}[subsection]
% \newtheorem{twr}[deff]{Twierdzenie}
% \newtheorem{lem}{Lemat}
% \newtheorem{uwaga}[deff]{Uwaga}
% \newtheorem{alg}[deff]{Algorytm}


\newtheorem{deff}{Definition}[subsection]
\newtheorem{twr}[deff]{Theorem}
\newtheorem{lem}{Lemma}
\newtheorem{uwaga}[deff]{Remark}
\newtheorem{alg}[deff]{Algorithm}

%\newcommand{\e}[0]{\mathbf{e}}
%\newcommand{\s}[0]{\mathbf{s}}
%\newcommand{\PP}[0]{\mathbf{P}}
%\newcommand{\E}[0]{\mathbb{E}}
 
\renewcommand\theequation{\thesection.\arabic{equation}}
%\setlength{\textwidth}{16cm}
%\setlength{\oddsidemargin}{3cm}
%\setlength{\evensidemargin}{3cm}
%\setlength{\hoffset}{-1in} %


\addtolength{\voffset}{-2.0cm}
\addtolength{\hoffset}{-1.0cm}
\addtolength{\textwidth}{2.0cm}
\addtolength{\textheight}{3.0cm}



\setlength{\fboxrule}{11cm}

\usepackage{fancyhdr}
\pagestyle{fancy}
\renewcommand\headrulewidth{0pt}
\fancyhead[LE,LO,RE,RO]{}
\fancyfoot[LE,LO]{\tiny {\tt\jobname    }}
\fancyfoot[RE,RO]{\tiny {\tt \leftmark     }}
%\rightmark - \leftmark --
\fancyfoot[c]{\thepage   }

\newcount\cnt
\def\nn {{
{\message{-\the\cnt-}}\par
\bf \the\cnt .\   \global\advance\cnt by 1}}
\def\nx {{
{\message{-\the\nu-}}\hskip 10truept
\bf \the\cnt .\   \global\advance\cnt by 1}}
\def\ny {{
{\message{-\the\cnt-}}\par\noindent
\bf \the\cnt .\   \global\advance\cnt by 1}}
\def\nz {{\hskip -2truept
{\message{-\the\cnt-}}
\bf \the\cnt .\   \global\advance\cnt by 1}}

\def\bp {\bigskip\par}

\usepackage{setspace}

\parindent 0pt

\everymath{\displaystyle}

\input{setup/packages2}
\input{setup/macros2}
\begin{document}
 
\noindent
 {
\setlength\fboxsep{4pt}%
 \setlength\fboxrule{2pt}%
 \fbox{%
  \begin{tabular}{lcr}
\multicolumn{2}{l}{\bf  Simulations and algorithmic applications of Markov chains} &     2024/25   \\ \\
\multicolumn{3}{c}{   List nr 2} \\
\multicolumn{2}{l}{ Paweł Lorek} &  \\  
 \end{tabular}%
}} \bigskip\bigskip
\par \bigskip

%\setstretch{1}
 
  

\begin{enumerate}

\item Let $X$ be an irreducible and aperiodic Markov chain on  $\E=\{\e_1,\ldots,\e_M\}$ 
with a transition matrix  $\PP$.
Define 
$$T_{i,j}=\min\{n\geq 1: X_n=\e_j\ | X_0=\e_i\}$$
(with convention $T_{i,j}=\infty$ if chain never reaches $\e_j$).
We show at the lecture that 
$\forall(\e_i,\e_j\in\E) \ \tau_{i,j}=ET_{i,j}<\infty$. We also defined
$$\rho(\e_i)=\sum_{n=0}^\infty Pr(X_n=\e_i, T_{1,1}\geq n)$$
and showed that  $\pi(\e_j):={\rho(\e_j)\over \tau_{1,1}}$ fulfills
\begin{equation}\label{eq:pi}
\pi(\e_j)=\sum_{i=1}^M \pi(\e_i)\PP(\e_i,\e_j)
\end{equation}
for $\e_j\neq \e_1$. Show that (\ref{eq:pi}) holds also for  $\e_j=\e_1$.


\item For $X$ from previous exercise let  $N_i$ be a number 
of visits in  $\e_i$ between visits in  $\e_1$
(for $i=1$ we do not count first visit, i.e., $N_1=1$). Show that 
$$EN_i=\rho(\e_i).$$

\item 
For Exercises   7, 8, 9 b) from List 1 show valid initialization and update functions 
(in case of Exercise 8 show at least two different initialization and update functions).

\item 
The total variation distance between two measures  $\nu$ and $\mu$ on $\E$ 
was defined as $d_{TV}(\mu,\nu)={1\over 2}\sum_{\e\in\E}|\mu(\e)-\nu(\e)|$.
Show that 
$$d_{TV}(\mu,\nu)=  \sum_{\e: \mu(\e)>\nu(\e)} (\mu(\e)-\nu(\e))=\sum_{\e: \nu(\e)>\mu(\e)} (\nu(\e)-\mu(\e)).$$


\item
Show that 
$$ d_{TV}(\mu,\nu)=\max_{A\subseteq \E} |\mu(A)-\nu(A)|,$$
where $\mu(A):=\sum_{\e\in A}\mu(A)$.

\item
Provide an example of a transition matrix $\PP$ of size $3\times 3$ for which there exist
at least two different distributions $\pi$ such that  $\pi\PP=\pi$.
Can you show such a matrix $\PP$ for which $\pi\PP=\pi$ holds for any distribution $\pi$?

\item
Let  $X=\{X\}_{n\geq 0}$ be a (time homogeneous) Markov chain on $\E=\{\e_1,\ldots,\e_M\}$.
Show that for integers  $a,a'>0$ and any  $\e\in\E$ we have
$$P(X_a=\e, X_{a+a'}=\e | X_0=\e) = P(X_{a'}=\e |X_0=\e)\cdot P(X_{a}=\e |X_0=\e).$$

\item(Ehrenfest Urn Model) Any out of $N$ particles 
may be either in a container $A$ or in a container $B$. Assume that at step $k\geq 0$ there are 
$X_k=i$ particles in a container $A$. Then, one particle is chosen uniformly at random 
and is relocated to the other container. Thus,  at step $k+1$, the state $X_{k+1}$ is either $i-1$ or $i+1$.
The state space is $\E=\{0,1,\ldots,N\}$. Provide a transition matrix $\PP$ and find its stationary distribution.


\item 
Referring to the previous exercise: Assume we have $N$ particles we place each one independently
either in container $A$ or $B$ (i.e., with probability 1/2).
What is a distribution of number of particles in container $A$ after such operation?

\item(Birth and death chain with two reflecting barriers)
Let  $X$ be a Markov chain on $\E=\{0,1,\ldots,N\}$ with following transition matrix:
$$\PP=\left[
\begin{array}{ccccccccc}
0 & 1 & 0 & 0 & 0  & 0 &\ldots & 0 \\[5pt]
q_1 & r_1 & p_1 & 0 & 0 & 0 &\ldots & 0 \\[5pt]
0 & q_2 & r_2 & p_2 & 0 & 0 &\ldots & 0\\[5pt]
  &      & \ddots& \ddots& \ddots\\[5pt]
0 & 0 & 0 & q_i & r_i & p_i &0 \ldots &0 \\[5pt]
 &  & &      & \ddots& \ddots& \ddots\\[5pt]
 0&\ldots &  & &      & q_{N-1} & r_{N-1} & p_{N-1} \\[5pt]
 0&\ldots &  & &      & 0 & 1 & 0 \\[5pt]
    \end{array}\right],$$
    where for any  $i\in\{1,\ldots,N-1\}$: $p_i>0, q_i>0, p_i+q_i+r_i=1$. 
    Find its stationary distribution.

\item
Find the stationary distribution of a Markov chain $X$ on  $\E=\{1,2,3\}$ with transition matrix
$$\PP=\left[
\begin{array}{ccccccccc}
1-\alpha & \alpha & 0 \\
0  &1-\beta &\beta  \\
\gamma&0&1-\gamma\\
    \end{array}\right],$$
where $\alpha,\beta,\gamma\in(0,1)$.

\item(\textsc{Top-To-Random} card shuffling) Show from definition that 
$\pi(\sigma)={1/n!}$ for any $\sigma\in\E=\mathcal{S}_n$ (all permutations of $\{1,\ldots,n\}$) 
is the stationary distribution \textsc{Top-To-Random} card shuffling.

\item
For \textsc{Top-To-Random} card shuffling of $n$ cards show smallest possible $k_0$ such that
it is possible to move from any permutation to any other, i.e.,
$$k_0=\argmin\{t:   \forall (\sigma_1,\sigma_2 \in\mathcal{S}_n)\quad  \Prob(X_k=\sigma_2 | X_0=\sigma_1)>0\}.$$
% 
% $$\forall (\sigma_1,\sigma_2 \in\mathcal{S}_n): \qquad \Prob(X_k=\sigma_2 | X_0=\sigma_1)>0.$$

\item 
Referring to the previous exercise: Assume the chain starts in identity permutation
(i.e., $\Prob(X_0=(1,\ldots,n))=1$). Show that the distribution of a chain at time $k_0$ is not uniform.

For  $n=3$ compute  $d_{TV}(\mu^{k_0},\mathcal{U}(\mathcal{S}_3),$ where 
$\mathcal{U}(\mathcal{S}_3)$ is a uniform distribution on $\mathcal{S}_3$.

\item
Let $X=\{X\}_{k\geq 0}$ be a Markov chain with a transition matrix $\PP^X$ on $\E=\{\e_1,\ldots,\e_M\}$.

We construct a sequence $Y=\{Y\}_{k\geq 0}$ based on $X$ in such a way that we record only 
\textsl{new} values of $X_n$.
I.e., if for some number of steps chain $X$ stays in the same state, value of $Y$ does not change,
it waits till $X$ will change. For example:
$$
\begin{array}{ccccccccccccccccccccccc}
 n = & 0 & 1 & 2 & 3 & 4 & 5 & 6 & 7 & 8 & 9 & 10\\[5pt]
 X_n = &\e_1 & \e_1 &\e_1 & \e_2 &\e_2 & \e_1 & \e_3& \e_3& \e_3& \e_2 &\e_1\\
 & Y_0=\e_1& & & Y_1=\e_2 & & Y_2=\e_1 & Y_3=\e_3& & & Y_4=\e_2 & Y_5=\e_1\\
\end{array}
$$
Show that $\{Y\}_{n\geq 0}$ is a Markov chain, provide its transition probabilities in terms 
of transition probabilities of $X$.

\item

Let $X=\{X\}_{n\geq 0}$ and $Y=\{Y\}_{n\geq 0}$ be two Markov chains 
on the same state space  $\E=\{1,2\}$ with transition matrix
$$\PP=\left[
\begin{array}{ccccccccc}
1-\alpha & \alpha \\
\beta & 1-\beta \\
\end{array}\right],$$
where $\alpha,\beta\in(0,1)$. Assume that  $X_0=1, Y_0=2$ and they both move according to $\PP$ independently.
Let $T$ be the first time the chains meet, i.e., 
$$T=\argmin_{k}\{X_k=Y_k\}.$$
Provide a distribution of $T$.



\item 
Let   $X=\{X\}_{n\geq 0}$ be a Markov chain with two 
stationary distributions  $\pi$ and $\pi'$. Show that then
$p\pi+(1-p)\pi'$ for any $p\in(0,1)$ is also its stationary distribution.

\item Consider the following Markov chain $X$. Let 
$a,b\geq 2$ be two integer numbers. The state space is:
    $$\E=\{(x,y) : x\in\{0,\ldots, a-1\}, y\in\{0,\ldots,b-1\}\}.$$
     
   The dynamics of the chain is as follows. Being in $(x,y)$ at step  $k$,  in the next step the 
   chain is either at
   \begin{itemize}
    \item[--]
    $((x+1) \textrm{ mod } a,y)$ with probability  ${1\over 2}$ or in a state
    \item[--] 
    $(x,(y+1) \textrm{ mod } b)$ with probability ${1\over 2}$.
    
   \end{itemize}

\begin{itemize}
 \item[a)] Show that $X$ is irreducible.
 \item[b)] Show that $X$ is aperiodic iff $\gcd(a,b)=1$
\end{itemize}
  
\end{enumerate}
 




\end{document}
