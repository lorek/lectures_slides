\documentclass[a4paper,12pt]{article}
\usepackage[utf8]{inputenc}
\usepackage{enumitem}
\usepackage[T1]{fontenc}
\usepackage{amssymb}
\usepackage{graphicx}
\usepackage{pstricks}
 \usepackage{amsthm}
 
%\usepackage[nohead]{geometry}
%\usepackage{amsmath,amsthm,amssymb,euler}

% \newtheorem{deff}{Definicja}[subsection]
% \newtheorem{twr}[deff]{Twierdzenie}
% \newtheorem{lem}{Lemat}
% \newtheorem{uwaga}[deff]{Uwaga}
% \newtheorem{alg}[deff]{Algorytm}
 \usepackage{algorithm}
\usepackage{algorithmic}
%Change: REQUIRE -> Input and ENSURE -> Output
%\renewcommand{\algorithmicrequire}{\textbf{Input:}}
%\renewcommand{\algorithmicensure}{\textbf{Output:}}


\newtheorem{deff}{Definition}[subsection]
\newtheorem{theorem}{Theorem}
\newtheorem{lem}{Lemma}
\newtheorem{uwaga}[deff]{Remark}
\newtheorem{alg}[deff]{Algorithm}

%\newcommand{\e}[0]{\mathbf{e}}
%\newcommand{\s}[0]{\mathbf{s}}
%\newcommand{\PP}[0]{\mathbf{P}}
%\newcommand{\E}[0]{\mathbb{E}}
 
\renewcommand\theequation{\thesection.\arabic{equation}}
%\setlength{\textwidth}{16cm}
%\setlength{\oddsidemargin}{3cm}
%\setlength{\evensidemargin}{3cm}
%\setlength{\hoffset}{-1in} %


\addtolength{\voffset}{-2.2cm}
\addtolength{\hoffset}{-1.0cm}
\addtolength{\textwidth}{2.0cm}
\addtolength{\textheight}{3.2cm}



\setlength{\fboxrule}{11cm}

\usepackage{fancyhdr}
\pagestyle{fancy}
\renewcommand\headrulewidth{0pt}
\fancyhead[LE,LO,RE,RO]{}
\fancyfoot[LE,LO]{\tiny {\tt\jobname    }}
\fancyfoot[RE,RO]{\tiny {\tt \leftmark     }}
%\rightmark - \leftmark --
\fancyfoot[c]{\thepage   }

\newcount\cnt
\def\nn {{
{\message{-\the\cnt-}}\par
\bf \the\cnt .\   \global\advance\cnt by 1}}
\def\nx {{
{\message{-\the\nu-}}\hskip 10truept
\bf \the\cnt .\   \global\advance\cnt by 1}}
\def\ny {{
{\message{-\the\cnt-}}\par\noindent
\bf \the\cnt .\   \global\advance\cnt by 1}}
\def\nz {{\hskip -2truept
{\message{-\the\cnt-}}
\bf \the\cnt .\   \global\advance\cnt by 1}}

\def\bp {\bigskip\par}

\usepackage{setspace}

\parindent 0pt

\everymath{\displaystyle}

\input{/home/lorek/Dropbox/LaTeX/setup/packages2_noalg}
\input{/home/lorek/Dropbox/LaTeX/setup/macros2}
 
\begin{document}
 
\noindent
 {
\setlength\fboxsep{4pt}%
 \setlength\fboxrule{2pt}%
 \fbox{%
  \begin{tabular}{lcr}
\multicolumn{2}{l}{\bf  Simulations and algorithmic applications of Markov chains} &     2024/25  \\ \\
\multicolumn{3}{c}{   List nr 8} \\
\multicolumn{2}{l}{ Paweł Lorek} &  \\  
 \end{tabular}%
}}
\bigskip\bigskip
  \begin{center}
   \textbf{Metropolis Algorithm}
  \end{center}
  
\par \bigskip
Given a distribution $\pi$ on (usually huge) state space $\E=\{\e_1,\ldots,\e_M\}$, the aim is to sample from $\pi$ (often called a \textbf{target distribution}). 


The goal of Metropolis algorithm: construct a Markov chain whose stationary distribution is $\pi$.

We can think of $\E$ as of set of vertices $v_i=\e_i$ and of Markov chain as a random walk on this graph. 
Let $\Q$ be a matrix of size $M\times M$, denote $\Q=[q_{i,j}]_{i,j=1}^M\equiv [q(\e_i,\e_j)]$.
Assume that $\Q$ is \textbf{symmetric}. For example, we can introduces some vertices $K$, i.e.,
there is an edge between $v_i$ and $v_j$ (often we say they are neighbours) if  $(v_i,v_j)\in K$.
\underline{As an example} we may take the following random walk on $\E$: let $d=\max_{v} \textrm{deg}(v)$ and 

$$q(v,v')=\left\{
\begin{array}{llllcccccccc}
{1\over 2d} & \textrm{if }   (v,v')\in K,\\[12pt]
1-{\textrm{deg}(v)\over 2d}  & \textrm{if }   v=v',\\[12pt]
0 & \textrm{otherwise}.
    \end{array}\right.$$

Note that a random walk with matrix $\Q$ has uniform stationary distribution (since $\Q$ is symmetric).
Often $\Q$ is called a \textbf{proposal distribution}.
\medskip\par 
\textbf{Metropolis algorithm} is the following modification of $\Q$. Consider $X$ with t.m.

$$\PP(v,v')=\left\{
\begin{array}{llccccc}
\Q(v,v')\cdot\min\left(1,{\pi(v')\over \pi(v)}\right) & \textrm{if }  v\neq v'\\[12pt]
1-\sum_{w\neq v} \Q(v,w)\cdot\min\left(1,{\pi(w)\over \pi(v)}\right)   & \textrm{if }   v=v',\\[12pt]
0 & \textrm{otherwise}.
    \end{array}\right.$$
On lecture we have shown that $\pi$ is the stationary distribution of $X$.
\medskip\par 
It is very convenient to write it as an algorithm:
     
     
 
\begin{algorithm}[H]
\caption{The  Metropolis algorithm for sampling for a distribution  $\pi$ on $\E$ 
using a symmetric regular matrix $\Q$ (1 step)} 
\label{alg:metropolis}
\begin{algorithmic}[1] 
%\REQUIRE Sample size: $n$, number of stratas: $m$
 
%\FOR{$k=1$ to $[n/m]$}
\STATE assume that  $X_k=\e$
\STATE generate $Z$ according to a distribution  $\Q(\e,\cdot)\equiv [\Q(\e,\e_1),\ldots,\Q(\e,\e_M)].$
\STATE assume $Z=\e'$
\STATE set   $\alpha=\min(1,\pi_{\e'}/\pi_\e)$ % \pl{ $\min(1,\pi_X,\pi_i)$?}
\STATE generate $U\sim \mathcal{U}(0,1)$
\STATE \textbf{if} $U\leq \alpha$ \textbf{then} $X_{k+1}=\e'$
\STATE $\qquad \qquad $ \textbf{else}  $X_{k+1}=X_k$
\RETURN $Y_{k+1}$.
\end{algorithmic}
\end{algorithm}


\begin{enumerate}
 \item Let $\E=\mathcal{S}_n$ be a set of all permutations of $n$ elements ('cards').
 %Let $\sigma_{id}=(1,2,\ldots,n)$ be an identity permutation.
 Consider the following proposal distribution 
 $$\Q(\sigma,\sigma')=
 \left\{ 
 \begin{array}{llllllll}
  {1\over 2} & \textrm{if } \sigma'=\sigma, \\[12pt] 
  {1\over 2 {n\choose 2}} & \textrm{if } \sigma'\in\mathcal{N}(\sigma), \\[12pt] 
  0  & \textrm{otherwise}, 
 \end{array}
\right.$$
where $\mathcal{N}(\sigma)$ are all the permutations (\textsl{neighbours}) obtained from $\sigma$ by swapping two elements.

\smallskip\par 
Provide a Markov chain, Metropolis algorithm with above proposal distribution, for sampling 
from target distribution:
$$\pi(\sigma)={\theta^{f(\sigma)}\over z},$$
where $Z=\sum_{\sigma} \theta^f(\sigma)$ is a normalization constant and $f(\sigma)$ is defined as follows:
$$f(\sigma)=\sum_{k=1}^n k\cdot \sigma_k$$
(For example, $f((3,5,4,1,2))=1\cdot 3 + 2\cdot 5+3\cdot 4 + 4\cdot 1 + 5\cdot 2 = 39$).
  
\item (\textbf{Graph coloring}). Let $G=(V,K)$ be a graph and $r\geq 2$ be an integer.
A proper coloring of $G$ is an assignment of color $S=\{1,\ldots,r\}$ to vertices
with the property that no two neighbours have the same color. Let $\Lambda$ be a set of such proper colorings,
i.e.,
$$\Lambda=\{\e: \e(v)\neq \e(w) \textrm{ if } (v,w)\in K\}. $$
Let $\pi$ be a uniform distribution on $\Lambda$, i.e.,
$$\pi(\e)=
\left\{
\begin{array}{llll}
{1\over z} & \textrm{ if } \e\in\Lambda,\\[12pt]
0 & \textrm{otherwise}.
\end{array}
\right.$$
Provide a Metropolis algorithm for sampling from $\pi$.

\item (\textbf{Ising model}). Let $\mathcal{G}=(\mathcal{V},\mathcal{K})$ be a graph with some graph with vertices $v_1,\ldots,v_n$. 
By state (often called \textbf{configuration} in this context) we mean assigning values $+1$ or $-1$ to each vertex. In other words, $\e\in\E=\{-1,+1\}^\mathcal{V}$. For a given configuration we define energy function
$$H(\e)=\sum_{(v,v')\in \mathcal{K}} \e(v)\e(v').$$
One calls values $\pm 1$ \textsl{spins}. Provide a Metropolis algorithm for sampling
from target distribution 
$$\pi(\e)={1\over z} e^{-\beta H(\e)},$$
where $z=\sum_{z\in\E}e^{-\beta H(\e)}$ is a normalization constant.
Use the following proposal distribution:
$$\Q(\e,\e')=
\left\{
\begin{array}{llll}
{1\over n} & \textrm{if } d(\e',\e)=1,\\[12pt]
0 & \textrm{otherwise},
\end{array}
\right.$$ 
where $d(\e',\e)$ is a number of vertices where $\e$ and $\e'$ differ
$$d(\e',\e)=\#\{v: \e'(v)\neq \e(v)\}.$$
In other words: given $\e'$ we sample a candidate in Metropolis algorithm simply chosing a vertex $v$ uniformly at random (with probability $1/n$, we have $n$ vertices) and swapping its value $-1 \leftrightarrow +1$.
\smallskip\par 
Note: We have two graphs here. Graph $\mathcal{G}$ used to define the target distribution $\pi$. 
The Markov chain will be a chain on configurations $\E=\{-1,+1\}^\mathcal{V}$. Here there is an edge between 
$\e$ and $\e'$ defined by $\Q$: the states are neighbours if they differ exactly at one vertex.



 
\end{enumerate}

\newpage
In a case when matrix $\Q$ is not symmetric, but it has the property that $\Q(\e,\e')>0 \iff \Q(\e',\e)>0$, we may use Metropolis-Hasting algorithm:
\begin{algorithm}[H]
\caption{The  Metropolis-Hasting algorithm for sampling for a distribution  $\pi$ on $\E$
using a  regular matrix $\Q$ (1 step)}
\label{alg:metropolis-hasting}
\begin{algorithmic}[1]
%\REQUIRE Sample size: $n$, number of stratas: $m$

%\FOR{$k=1$ to $[n/m]$}
\STATE assume that  $X_k=\e$
\STATE generate $Z$ according to a distribution  $\Q(\e,\cdot)\equiv [\Q(\e,\e_1),\ldots,\Q(\e,\e_M)].$
\STATE assume $Z=\e'$
\STATE set   $$\alpha=\min\left(1,{\pi(\e')\Q(\e',\e) \over \pi(\e)\Q(\e,\e')}\right)$$ % \pl{ $\min(1,\pi_X,\pi_i)$?}
\label{alg:mh_line_alpha}
\STATE generate $U\sim \mathcal{U}(0,1)$
\STATE \textbf{if} $U\leq \alpha$ \textbf{then} $X_{k+1}=\e'$
\STATE $\qquad \qquad $ \textbf{else}  $X_{k+1}=X_k$
\RETURN $Y_{k+1}$.
\end{algorithmic}
\end{algorithm}

Note that Algorithm \ref{alg:metropolis-hasting} differs from Algorithm \ref{alg:metropolis} only
in  computating $\alpha$ in line \ref{alg:mh_line_alpha}.

\begin{enumerate}[resume]
  \item  Modify line \ref{alg:mh_line_alpha} of Algorithm \ref{alg:metropolis-hasting}, set
  $$\alpha={\pi(\e')\Q(\e',\e) \over \pi(\e')\Q(\e',\e) +\pi(\e)\Q(\e,\e') }.$$
  Show that the resulting algorithm (\textsl{Barker's algorithm})  yields a Markov chain with stationary distribution $\pi$.

\item Let $G=(V,K)$ be a connected graph.
Provide a Metropolis(-Hasting) algorithm for sampling from distribution ($z$ is a normalization constant)
$$\pi(v)={1\over z}{1\over (\rm{deg}(v))^2}.$$
Consider two different proposal distributions represented by $\Q_1$ and $\Q_2$ respectively.
\begin{enumerate}
 \item[a)]
 $$\Q_1(v,v')=
\left\{
\begin{array}{llll}
{1\over 2m} & \textrm{if } (v,v')\in K,\\[14pt]
1-{\textrm{deg}(v)\over 2m}  & \textrm{if } v=v',\\
0 & \textrm{otherwise},
\end{array}
\right.$$
where $m= \max_{w} \textrm{deg}(w)$.
 \item[b)]
$$\Q_2(v,v')=
\left\{
\begin{array}{llll}
{1\over 2\textrm{deg}(v)} & \textrm{if } (v,v')\in K,\\[14pt]
{1\over 2}  & \textrm{if } v=v',\\[14pt]
0 & \textrm{otherwise}.
\end{array}
\right.$$
\end{enumerate}





\end{enumerate}

\end{document}
