\documentclass[a4paper,12pt]{article}
\usepackage[utf8]{inputenc}
\usepackage[T1]{fontenc}
\usepackage{amssymb}
\usepackage{graphicx}
\usepackage{pstricks}
 \usepackage{amsthm}
 
%\usepackage[nohead]{geometry}
%\usepackage{amsmath,amsthm,amssymb,euler}

% \newtheorem{deff}{Definicja}[subsection]
% \newtheorem{twr}[deff]{Twierdzenie}
% \newtheorem{lem}{Lemat}
% \newtheorem{uwaga}[deff]{Uwaga}
% \newtheorem{alg}[deff]{Algorytm}


\newtheorem{deff}{Definition}[subsection]
\newtheorem{twr}[deff]{Theorem}
\newtheorem{lem}{Lemma}
\newtheorem{uwaga}[deff]{Remark}
\newtheorem{alg}[deff]{Algorithm}

%\newcommand{\e}[0]{\mathbf{e}}
%\newcommand{\s}[0]{\mathbf{s}}
%\newcommand{\PP}[0]{\mathbf{P}}
%\newcommand{\E}[0]{\mathbb{E}}
 
\renewcommand\theequation{\thesection.\arabic{equation}}
%\setlength{\textwidth}{16cm}
%\setlength{\oddsidemargin}{3cm}
%\setlength{\evensidemargin}{3cm}
%\setlength{\hoffset}{-1in} %


\addtolength{\voffset}{-2.0cm}
\addtolength{\hoffset}{-1.0cm}
\addtolength{\textwidth}{2.0cm}
\addtolength{\textheight}{3.0cm}



\setlength{\fboxrule}{11cm}

\usepackage{fancyhdr}
\pagestyle{fancy}
\renewcommand\headrulewidth{0pt}
\fancyhead[LE,LO,RE,RO]{}
\fancyfoot[LE,LO]{\tiny {\tt\jobname    }}
\fancyfoot[RE,RO]{\tiny {\tt \leftmark     }}
%\rightmark - \leftmark --
\fancyfoot[c]{\thepage   }

\newcount\cnt
\def\nn {{
{\message{-\the\cnt-}}\par
\bf \the\cnt .\   \global\advance\cnt by 1}}
\def\nx {{
{\message{-\the\nu-}}\hskip 10truept
\bf \the\cnt .\   \global\advance\cnt by 1}}
\def\ny {{
{\message{-\the\cnt-}}\par\noindent
\bf \the\cnt .\   \global\advance\cnt by 1}}
\def\nz {{\hskip -2truept
{\message{-\the\cnt-}}
\bf \the\cnt .\   \global\advance\cnt by 1}}

\def\bp {\bigskip\par}

\usepackage{setspace}

\parindent 0pt

\everymath{\displaystyle}

\input{setup/packages2}
\input{setup/macros2}
\begin{document}
 
\noindent
 {
\setlength\fboxsep{4pt}%
 \setlength\fboxrule{2pt}%
 \fbox{%
  \begin{tabular}{lcr}
\multicolumn{2}{l}{\bf  Simulations and algorithmic applications of Markov chains} &     2024/25   \\ \\
\multicolumn{3}{c}{   List nr 3} \\
\multicolumn{2}{l}{ Paweł Lorek} &  \\  
 \end{tabular}%
}} \bigskip\bigskip
\par \bigskip

%\setstretch{1}
 
  

\begin{enumerate}
\item A lonely chess king is standing on a standard $8\times  8$   board, who moves randomly into
one of the closest squares (the distribution is uniform).
\begin{itemize}
 \item  Is the Markov chain related to the random walk above aperiodic and/or irreducible?
 \item Find the stationary distribution of this chain.
\end{itemize}


\item 


Let $(X_0, X_1,\ldots)$ be a reversible Markov chain on a state space $\E$, with a transition matrix  $\PP$ and a reversible distribution $\pi$.
 Show that if the chain started with an initial
distribution $\mu=\pi$,  then for every  $n\in\mathbb{N}$  and every 
$\e_{i_0},\e_{i_1},\ldots,\e_{i_n}\in \E$  we have
$$P(X_0=\e_{i_0}, X_1=\e_{i_1}, \ldots, X_n=\e_{i_n}) = P(X_0=\e_{i_n}, X_1=\e_{i_{n-1}}, \ldots, X_n=\e_{i_0}).$$

\item 

Prove that if a transition matrix $\PP$ is \textsl{ doubly stochastic}
(i.e., the sum of elements in
each columns is also equal to 1), then the stationary distribution of the chain  is uniform.


\item 
Let  $\PP$ be the transition matrix of an ergodic chain  $X$ with the stationary distribution $\pi$. 
The transition matrix of the time-reversed chain $\tilde{X}$  s defined as  $\tilde{\PP}(\e_2,\e_1)={\pi(\e_1)\over \pi(\e_2)} \PP(\e_1,\e_2)$
(a chain is reversible  $\PP=\tilde{\PP}$). Show that $\pi$ is also a stationary distribution of   $\tilde{X}$.



\item 

Let $\PP$ be the transition matrix of an ergodic chain $X$ with the stationary distribution $\pi$.
 Define:  $\mathbf{M}:=\PP\cdot \tilde{\PP}$. Show that
\begin{itemize}
\item $\mathbf{M}$ is a stochastic matrix.
 \item $\pi$ is a stationary distribution of a chain with the transition matrix $\mathbf{M}$.  
 Is this chain reversible?

\end{itemize}


\item 
(Birth and death chain).
Let $X$ be  a Markov chain on a state space  $\E=\{0,1,\ldots,N\}$ with the transition matrix
$$\PP(i,j)=\left\{
\begin{array}{clll}
p_i & \textrm{if } j=i+1, \\[6pt]
q_i & \textrm{if } j=i-1, \\[6pt]
r_i & \textrm{if } j=i,
\end{array}
\right.
$$
where  $p_0>0, q_0=0, p_N=0, q_N>0$ and for all $i\in\{1,\ldots,N-1\}$: $p_i>0, q_i>0, p_i+q_i+r_i=1$.
Show that this chain is reversible. 
  
  
  \item In List nr 2, Exercise nr 16, independent chains $X=\{X\}_{n\geq 0}$ and $Y=\{Y\}_{n\geq 0}$  with transition matrix
$$\PP=\left[
\begin{array}{ccccccccc}
1-\alpha & \alpha \\
\beta & 1-\beta \\
\end{array}\right]$$
were considered. 
Assume that  $X_0=1, Y_0=2$ and they both move according to $\PP$, however this time 
they may be \textsl{dependent}. The task is, assuming additionally that $\alpha+\beta<1$,
to construct some (dependent) coupling $(X_k, Y_k)$ of the chains, so that the first time the chains meet 
$$T'=\argmin_k\{X_k=Y_k\}$$
is ``better'' than $T$ resulting from independent chains (i.e., $T$ from List 2, Exercise 16),
here better means $ET'<ET$ (compute those means). 
\begin{flushright}
 \tiny{\texttt{Hint: use the same update function for both chains}}
\end{flushright}

\item Continuing previous exercise: using your coupling and independent coupling (List 2, Exercise 16) provide upper bounds 
on $d_{TV}(\mu\PP^k,\pi)$ in both cases
\begin{enumerate}
 \item using Markov inequality,
 \item Chebyshev inequality.
\end{enumerate}






 

\end{enumerate}
 




\end{document}
