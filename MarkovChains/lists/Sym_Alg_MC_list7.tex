\documentclass[a4paper,12pt]{article}
\usepackage[utf8]{inputenc}
\usepackage{enumitem}
\usepackage[T1]{fontenc}
\usepackage{amssymb}
\usepackage{graphicx}
\usepackage{pstricks}
 \usepackage{amsthm}
 
%\usepackage[nohead]{geometry}
%\usepackage{amsmath,amsthm,amssymb,euler}

% \newtheorem{deff}{Definicja}[subsection]
% \newtheorem{twr}[deff]{Twierdzenie}
% \newtheorem{lem}{Lemat}
% \newtheorem{uwaga}[deff]{Uwaga}
% \newtheorem{alg}[deff]{Algorytm}


\newtheorem{deff}{Definition}[subsection]
\newtheorem{theorem}{Theorem}
\newtheorem{lem}{Lemma}
\newtheorem{uwaga}[deff]{Remark}
\newtheorem{alg}[deff]{Algorithm}

%\newcommand{\e}[0]{\mathbf{e}}
%\newcommand{\s}[0]{\mathbf{s}}
%\newcommand{\PP}[0]{\mathbf{P}}
%\newcommand{\E}[0]{\mathbb{E}}
 
\renewcommand\theequation{\thesection.\arabic{equation}}
%\setlength{\textwidth}{16cm}
%\setlength{\oddsidemargin}{3cm}
%\setlength{\evensidemargin}{3cm}
%\setlength{\hoffset}{-1in} %


\addtolength{\voffset}{-2.0cm}
\addtolength{\hoffset}{-1.0cm}
\addtolength{\textwidth}{2.0cm}
\addtolength{\textheight}{3.0cm}



\setlength{\fboxrule}{11cm}

\usepackage{fancyhdr}
\pagestyle{fancy}
\renewcommand\headrulewidth{0pt}
\fancyhead[LE,LO,RE,RO]{}
\fancyfoot[LE,LO]{\tiny {\tt\jobname    }}
\fancyfoot[RE,RO]{\tiny {\tt \leftmark     }}
%\rightmark - \leftmark --
\fancyfoot[c]{\thepage   }

\newcount\cnt
\def\nn {{
{\message{-\the\cnt-}}\par
\bf \the\cnt .\   \global\advance\cnt by 1}}
\def\nx {{
{\message{-\the\nu-}}\hskip 10truept
\bf \the\cnt .\   \global\advance\cnt by 1}}
\def\ny {{
{\message{-\the\cnt-}}\par\noindent
\bf \the\cnt .\   \global\advance\cnt by 1}}
\def\nz {{\hskip -2truept
{\message{-\the\cnt-}}
\bf \the\cnt .\   \global\advance\cnt by 1}}

\def\bp {\bigskip\par}

\usepackage{setspace}

\parindent 0pt

\everymath{\displaystyle}

\input{/home/lorek/Dropbox/LaTeX/setup/packages2}
\input{/home/lorek/Dropbox/LaTeX/setup/macros2}
 
\begin{document}
 
\noindent
 {
\setlength\fboxsep{4pt}%
 \setlength\fboxrule{2pt}%
 \fbox{%
  \begin{tabular}{lcr}
\multicolumn{2}{l}{\bf  Simulations and algorithmic applications of Markov chains} &    2024/25  \\ \\
\multicolumn{3}{c}{   List nr 7 } \\
\multicolumn{2}{l}{ Paweł Lorek} &  \\  
 \end{tabular}%
}}
\bigskip\bigskip
  \begin{center}
   \textbf{Strong Stationary Times}
  \end{center}
  
  \section{Matching in graph}
  Recall, the algorithm for matching (running time -- SST): Initailly color all pairs black.
  \begin{itemize}
   \item \textbf{If} $L_k=R_k$ and chosen pair is black -- color it red
   \item \textbf{If} $L_k\neq R_k$ and exactly one of pairs is black -- color it red
   \item \textbf{If} all pairs red then \textbf{Stop}
  \end{itemize}
We showed that its running time $T$ can be decomposed as
$$T=\sum_{i=0}^{n-1} X_i,$$
where $X_i$ is $Geo(p_i)$, $i$ pairs are red, coloring next one happens with probability
$$p_i={n-i \over n^2}+2{i\over n}{(n-i)\over n}={(n-i)(2i+1)\over n^2}$$
We have
$$ET=\sum_{i=0}^{n-1} EX_i =\sum_{i=0}^{n-1} {1\over p_i} = n^2 \sum_{i=0}^{n-1} {1\over (n-i)(2i+1)}
=n^2 \sum_{i=1}^{n} {1\over (n-i+1)(2i-1)}$$
Note:
$$
{2\over 2i-1} + {1\over n-i+1} = {2(n-i+1)+2i-1\over (n-i+1)(2i-1)}={2n\over (n-i+1)(2i-1)},
$$
thus
$${1\over (n-i+1)(2i-1)}={1\over 2n} {2n\over (n-i+1)(2i-1)}={1\over 2n} \left({2\over 2i-1} + {1\over n-i+1}\right)$$
and
$$ET=n^2 \sum_{i=1}^{n} {1\over (n-i+1)(2i-1)}={n\over 2} \sum_{i=1}^{n} \left({2\over 2i-1} + {1\over n-i+1}\right)
$$
Seperately:
$$\sum_{i=1}^{n}  {1\over n-i+1}=\sum_{i=1}^n {1\over i} = H_n \approx \log n$$
For the second term we may approximate
$$\sum_{i=1}^{n} {2\over 2i-1}\approx \sum_{i=1}^{n} {2\over 2i} =    H_n \approx \log n$$
% or more accurately
% $$\sum_{i=1}^{n} {2\over 2i-1}=2\left(\sum_{i=1}^{2n-1} {1\over i}-\sum_{i=1}^{n-1} {1\over 2i}\right)
% =2\left(H_{2n-1}-{1\over 2}H_{n-1}\right)=2H_{2n-1}-H_{n-1}\approx 2\log(2n)-\log n=\log(4n^2)=\log(4n)$$

Finally,
$$ET\approx {n\over 2} \cdot 2\log n = n\log n$$


\par \bigskip
\section{Random-to-random transposition}
We have $n$ cards. At step $k$ we choose independently $L_k\sim\mathcal{U}(\{1,\ldots,n\})$ and
 $R_k\sim\mathcal{U}(\{1,\ldots,n\})$ and we swap cards  numbered $L_k$ and $R_k$.
 \medskip\par
 NOTE: if \textsl{``we swap cards  at positions $L_k$ and $R_k$''} \textsl{``we swap cards  numbered $L_k$ and $R_k$''},
 this describes the same Markov chain.

 \medskip\par
 You may use:
 $$H_n=\sum_{i=1}^n {1\over i} \approx \log n ,\qquad H_n^{(2)}=\sum_{i=1}^n {1\over i^2}\approx {\pi^2\over 6},
 \qquad \sum_{i=1}^{n/2}{n^2\over (n-i)^2}\leq 2n.$$

 \medskip\par


Below, different marking schemes are described.  Let $T$ be the time till all cards are colored red.
For a given scheme:
\begin{itemize}
 \item[a)] Show that  $T$ is SST.
 \item[b)] Compute $ET$ and $VarT$ (results $O(\cdot)$ are enough)
 \item[c)] Using Chebyshev inequality indicate $k$ for which $sep(\mu\PP^k,\pi)$ is small\par
 (e.g., as we had for Top-2-Random, we had $sep(\mu\PP^k,\pi)\leq c$ for $k=n\log n+cn$)
\end{itemize}


\begin{enumerate}
\item Simple marking scheme (yields result $ET=O(n^2)$):

\begin{itemize}
 \item If $L_k$ and $R_k$ are black, color $R_k$ red.
 \item If all cards are red, then STOP.
\end{itemize}



\item Broder's marking scheme (yields result $ET = 2n\log n +o(n\log n)$):

\begin{itemize}
 \item If $L_k=R_k$ is black , color the card $L_k=R_k$ red.
 \item If $L_k\neq R_k$ and $L_k$ is red, color $R_k$ red.
 \item If all cards are red, then STOP.
\end{itemize}




\item(*) Matthew's marking scheme (yields result $ET = {1\over 2}n\log n +o(n\log n)$):

\begin{itemize}
 \item If $k\leq \lfloor{n/2}\rfloor$: If $L_k$ and $R_k$ are black, color $R_k$ red.
 \item If $k>\lfloor{n/2}\rfloor$

\begin{itemize}
  \item If one $L_k, R_k$ is red and the other is black, color the other red.
  \item If $L_k=R_k$ is black , color the card $L_k=R_k$ red.
  \item Let $\phi$ be some an injection from all black cards into pairs of red cards.
  If $(L_k, R_k)=\phi(i)$, color the card $i$ red.
\end{itemize}

\end{itemize}






\end{enumerate}



\end{document}
