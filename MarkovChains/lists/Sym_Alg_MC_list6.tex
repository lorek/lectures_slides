\documentclass[a4paper,12pt]{article}
\usepackage[utf8]{inputenc}
\usepackage{enumitem}
\usepackage[T1]{fontenc}
\usepackage{amssymb}
\usepackage{graphicx}
\usepackage{pstricks}
 \usepackage{amsthm}
 
%\usepackage[nohead]{geometry}
%\usepackage{amsmath,amsthm,amssymb,euler}

% \newtheorem{deff}{Definicja}[subsection]
% \newtheorem{twr}[deff]{Twierdzenie}
% \newtheorem{lem}{Lemat}
% \newtheorem{uwaga}[deff]{Uwaga}
% \newtheorem{alg}[deff]{Algorytm}


\newtheorem{deff}{Definition}[subsection]
\newtheorem{theorem}{Theorem}
\newtheorem{lem}{Lemma}
\newtheorem{uwaga}[deff]{Remark}
\newtheorem{alg}[deff]{Algorithm}

%\newcommand{\e}[0]{\mathbf{e}}
%\newcommand{\s}[0]{\mathbf{s}}
%\newcommand{\PP}[0]{\mathbf{P}}
%\newcommand{\E}[0]{\mathbb{E}}
 
\renewcommand\theequation{\thesection.\arabic{equation}}
%\setlength{\textwidth}{16cm}
%\setlength{\oddsidemargin}{3cm}
%\setlength{\evensidemargin}{3cm}
%\setlength{\hoffset}{-1in} %


\addtolength{\voffset}{-2.0cm}
\addtolength{\hoffset}{-1.0cm}
\addtolength{\textwidth}{2.0cm}
\addtolength{\textheight}{3.0cm}



\setlength{\fboxrule}{11cm}

\usepackage{fancyhdr}
\pagestyle{fancy}
\renewcommand\headrulewidth{0pt}
\fancyhead[LE,LO,RE,RO]{}
\fancyfoot[LE,LO]{\tiny {\tt\jobname    }}
\fancyfoot[RE,RO]{\tiny {\tt \leftmark     }}
%\rightmark - \leftmark --
\fancyfoot[c]{\thepage   }

\newcount\cnt
\def\nn {{
{\message{-\the\cnt-}}\par
\bf \the\cnt .\   \global\advance\cnt by 1}}
\def\nx {{
{\message{-\the\nu-}}\hskip 10truept
\bf \the\cnt .\   \global\advance\cnt by 1}}
\def\ny {{
{\message{-\the\cnt-}}\par\noindent
\bf \the\cnt .\   \global\advance\cnt by 1}}
\def\nz {{\hskip -2truept
{\message{-\the\cnt-}}
\bf \the\cnt .\   \global\advance\cnt by 1}}

\def\bp {\bigskip\par}

\usepackage{setspace}

\parindent 0pt

\everymath{\displaystyle}

\input{setup/packages2}
\input{setup/macros2}
\begin{document}
 
\noindent
 {
\setlength\fboxsep{4pt}%
 \setlength\fboxrule{2pt}%
 \fbox{%
  \begin{tabular}{lcr}
\multicolumn{2}{l}{\bf  Simulations and algorithmic applications of Markov chains} &     2024/25  \\ \\
\multicolumn{3}{c}{   List nr 6} \\
\multicolumn{2}{l}{ Paweł Lorek} &  \\  
 \end{tabular}%
}} \bigskip\bigskip
\par \bigskip

%\setstretch{1}
  


\begin{enumerate}
 \item Once again consider a random walk on a circle: $\E=\{0,1,\ldots,n-1\}$, this time 
 with transition matrix:
 $$\PP=\left[
 \begin{array}{ccccccccc}
 1-p & p & 0 & 0  & \ldots & 0& 0 & 0\\[7pt]
 0 & 1-p & p & 0 & \ldots & 0 &0&0\\[7pt]
 0 & 0 & 1-p & p & 0 &\ldots & 0 &0\\[7pt]
   &    &     &   &   & \ddots & & \\[7pt]
 p & 0 & 0 & 0 & 0 &\ldots & 0 &1-p\\[7pt]
 \end{array}\right].$$
 In other words, the chain moves to the right (modulo $n$) with prob. $p$, with the remaining 
 probability it does not move.
 
 \par 
 
 Compute  $\mathbf{M}=\PP\tilde{\PP}$. Show that computing 
  $\gamma_C$ and $\gamma_P$ reduces to (previously computed) computing the constant for 
  symmetric random walk on circle. Recall the bounds, use them 
  to provide  bounds for  $d_{TV}(\delta_0\PP^k,\pi)$.
  
  
  \item Consider the following \textsl{broken} random walk on a circle $\E=\{0,1,\ldots,n-1\}$. 
  All the moves are like in a regular random walk on circle with $p=1/2$, however 
  instead of moving from last state $n-1$ to $0$, the chain stays at $n-1$ with prob. $1/2$.
  Thus, the transition matrix is following
 $$\PP=\left[
 \begin{array}{ccccccccc}
 0 & 1/2 & 0 & 0  & \ldots & 0& 0 & 1/2\\[7pt]
 1/2 & 0 & 1/2 & 0 & \ldots & 0 &0&0\\[7pt]
 0 & 1/2 & 0 & 1/2 & 0 &\ldots & 0 &0\\[7pt]
   &    &     &   &   & \ddots & & \\[7pt]
 0 & 0 & 0 & 0 & 0 &\ldots & 1/2&1/2\\[7pt]
 \end{array}\right].$$
 
 Show that the chain is ergodic and find its stationary distribution.
 
 \item Find transition of time reversal (matrix $\tilde{\PP}$) of the chain from previous Exerecise.
   
\end{enumerate}



\end{document}
