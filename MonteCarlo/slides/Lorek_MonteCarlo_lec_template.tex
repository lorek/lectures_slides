% This text is proprietary.
% It's a part of presentation made by myself.
% It may not used commercial.
% The noncommercial use such as private and study is free
% Sep. 2005 
% Author: Sascha Frank 
% University Freiburg 
% www.informatik.uni-freiburg.de/~frank/


\documentclass[aspectratio=169]{beamer}


%%% HANDOUT START
%  \documentclass[11pt,handout,aspectratio=169]{beamer}
%  \usepackage{pgfpages}
%  \pgfpagesuselayout{6 on 1}[a4paper]
%  \setbeamertemplate{footline}{% set footline options
%   $\qquad\qquad$\insertpagenumber   
%   }

 
%%% HANDOUT END

\usefonttheme{professionalfonts}
 
%\setbeamertemplate{page number in head/foot}[totalframenumber]
%\beamertemplatenavigationsymbolsempty

%\setbeamerfont{page number in head/foot}{size=\footnotesize}


%pdfnup --a4paper --keepinfo --nup 1x3 --frame true   --scale 0.92 --no-landscape --outfile handout.pdf slides.pdf


\setbeamercovered{transparent=10}


\usepackage{multirow}
% \usepackage{pgfpages}
% \mode<handout>{\setbeamercolor{background canvas}{bg=black!5}}
% \pgfpagesuselayout{4 on 1}[letterpaper,border shrink=2mm]

%\documentclass[handout]{beamer}
% \setbeamertemplate{bibliography entry title}{}
% \setbeamertemplate{bibliography entry location}{}
% \setbeamertemplate{bibliography entry note}{}
\setbeamertemplate{bibliography item}{\insertbiblabel}

\newcommand{\Prob}{{\rm I\hspace{-0.8mm}P}}

\newcommand{\p}[0]{\mathbf{p}}
\newcommand{\q}[0]{\mathbf{q}}

\newcommand{\e}[0]{\textbf{e}}
\newcommand{\PP}[0]{\mathbf{P}}
\newcommand{\II}[0]{\textbf{I}}
\newcommand{\Q}[0]{\mathbf{Q}}
\newcommand{\E}[0]{\mathbb{E}}
\newcommand{\X}[0]{\textbf{X}}
\newcommand{\Z}[0]{\textbf{Z}}
\newcommand{\s}[0]{\mathbf{s}}
\newcommand{\C}[0]{\textbf{C}}
\newcommand{\es}[0]{\mathbf{s}}
\newcommand{\trev}[0]{\overleftarrow}
\definecolor{mygray}{gray}{0.6}
\newcommand{\diag}{\mathbf{diag}}
\newcommand{\iii}[0]{\mathbf{i}}

%\dual to oznacznie dualnosci, oryginalnie *
%\newcommand{\dual}[0]{\circledast}
\newcommand{\dual}[0]{*}

%\monD to oznacznie tej drugiej monotonicznosci, czyli np. ^\monD-Mobius monotonicity
%\newcommand{\monD}[0]{\lozenge}
\newcommand{\monD}[0]{\downarrow}
\newcommand{\monU}[0]{\uparrow}

\usepackage{cancel} 

\usepackage{dsfont}
 
\newcommand{\indic}[0]{\mathds{1}}
\usepackage{pdfpages}
\usepackage{graphicx}
\usepackage{soul} 
\usepackage{tikz}
\usepackage[utf8]{inputenc}
\usepackage[T1]{fontenc}
\usetikzlibrary{arrows}




\setbeamertemplate{theorems}[numbered]

% Add support for \subsubsectionpage
%\def\subsubsectionname{\translate{Subsubsection}}
%\def\insertsubsubsectionnumber{\arabic{subsubsection}}
% \setbeamertemplate{subsubsection page}
% {
%   \begin{centering}
%     {\usebeamerfont{subsubsection name}\usebeamercolor[fg]{subsubsection name}\subsubsectionname~\insertsubsubsectionnumber}
%     \vskip1em\par
% %     \begin{beamercolorbox}[sep=4pt,center]{d part title}
% %       \usebeamerfont{subsubsection title}\insertsubsubsection\par
% %     \end{beamercolorbox}
%   \end{centering}
% }
% \def\subsubsectionpage{\usebeamertemplate*{subsubsection page}}


\AtBeginSection{\frame{\sectionpage}}
\AtBeginSubsection{\frame{\subsectionpage}}
\AtBeginSubsubsection{\frame{\subsubsectionpage}}
 
\usepackage{tikz}

\usepackage{pgfplots}
\pgfplotsset{compat=1.11}
 

 % FROM BOOK:
 


\usepackage{algorithm}

\usepackage{algpseudocode}
\usepackage{caption}
\usepackage{comment}
\usepackage{booktabs}


 \usepackage[T1]{fontenc}
%\usepackage[utf8]{inputenc}
%\usepackage[polish]{babel}

\usepackage{tikz}
\usetikzlibrary{arrows}
\usetikzlibrary{arrows.meta}
\usetikzlibrary{patterns}
%\usetikzlibrary{decorations.pathreplacing,calligraphy}
\usetikzlibrary{decorations.text}
\usetikzlibrary{decorations.markings}
\usetikzlibrary{decorations.pathreplacing}
\usetikzlibrary{arrows.meta,babel}
\usetikzlibrary{shapes.misc}

\usepackage{pgfplots}
%\pgfplotsset{compat=1.7}
\pgfplotsset{compat=1.16}
%\pgfplotsset{compat=1.18}
\usepgfplotslibrary{fillbetween}

\usepackage{subfig}

\usepackage{etoolbox}


% 2. Configuration Commands
% ============================
% Code listings (for \begin{lstlisting}...\end{lstlisting})
\usepackage{listings}
\lstset{
  basicstyle=\ttfamily\footnotesize,
  columns=fullflexible,
  keepspaces=true,
  showstringspaces=false,
  frame=none,
  numbers=none,
  breaklines=true,
}
%
% \renewcommand{\lstlistingname}{Python Listing} % Listing -> Algorithm
% \lstset{basicstyle=\fontsize{8}{12}\selectfont\ttfamily}
% \lstset{language=Python}
%
% \captionsetup[figure]{font=small} % Changes font size in figure captions
%
% \captionsetup[algorithm]{font=normal} % Changes font size in algorithm captions
% \AtBeginEnvironment{algorithmic}{\normalsize} % Changes font size inside algorithm body
%


% --- Bold (\bf) Commands ---


\newcommand{\bfM}{\mathbf{M}}
\newcommand{\bfU}{\mbox{\boldmath $U$}}
\newcommand{\bfu}{\mbox{\boldmath$u$}}
\newcommand{\bfv}{\mathbf{v}}
\newcommand{\bfw}{\mathbf{w}}
\newcommand{\bfx}{\mathbf{x}}
\newcommand{\done}{\textbf{\textcolor{red}{DONE }}}
\newcommand{\bfy}{\mbox{\boldmath$y$}}
\newcommand{\bfz}{\mathbf{z}}
\newcommand{\bfg}{\mbox{\boldmath$g$}}
\newcommand{\bfh}{\mbox{\boldmath$h$}}
\newcommand{\bfb}{\mbox{\boldmath$b$}}
\newcommand{\bfp}{\mathbf{p}}
\newcommand{\bfq}{\mathbf{q}}
\newcommand{\bfJ}{\mbox{\boldmath $J$}}
\newcommand{\bfP}{\mathbf{P}}
%\newcommand{\bfT}{\mbox{\boldmath $T$}}
\newcommand{\bfT}{\vec{T}}
\newcommand{\bfL}{\mathbf{L}}
\newcommand{\bfm}{\mbox{\boldmath$m$}}

\newcommand{\bfSigma}{\boldsymbol{\Sigma}}
\newcommand{\bfsigma}{\mbox{\boldmath $\sigma$}}
\newcommand{\bfmu}{\mbox{\boldmath $\mu$}}
\newcommand{\bfzero}{{\bf 0}}


\newcommand{\bfa}{\boldsymbol{a}}
\newcommand{\bfA}{\mathbf{A}}
\newcommand{\bfB}{\mathbf{B}}
\newcommand{\bbX}{\mathbb{X}}
\newcommand{\bfX}{\mathbf{X}}
\newcommand{\bfY}{\mathbf{Y}}
\newcommand{\bfZ}{\mathbf{Z}}
\newcommand{\bfV}{\mbox{\boldmath $V$}}
\newcommand{\bfC}{\mathbf{C}}
\newcommand{\bbR}{\mathbb{R}}
\newcommand{\bbN}{\mathbb{N}}
\newcommand{\bbZ}{\mathbb{Z}}
\newcommand{\bfS}{\mathbf{S}}
\newcommand{\bfD}{\mathbf{D}}
\newcommand{\bff}{\mathbf{f}}
\newcommand{\bfF}{\mathbf{F}}
\newcommand{\bfI}{\mathbf{I}}
\newcommand{\bfe}{\mathbf{e}}
\newcommand{\bfk}{\mbox{\boldmath$k$}}
\newcommand{\bfn}{\mbox{\boldmath$n$}}
\newcommand{\bfR}{\mathbf{R}}
\newcommand{\bfQ}{\mathbf{Q}}
\newcommand{\bfG}{\mbox{\boldmath $G$}}

\newcommand{\bfPi}{\mbox{\boldmath $\Pi$}}
\newcommand{\bfPhi}{\mbox{\boldmath $\Phi$}}
\newcommand{\bfPsi}{\mbox{\boldmath $\Psi$}}
\newcommand{\bfLambda}{\mbox{\boldmath $\Lambda$}}
\newcommand{\bflambda}{\mbox{\boldmath $\lambda$}}

\newcommand{\bfgamma}{\mbox{\boldmath $\gamma$}}
\newcommand{\bfalpha}{\mbox{\boldmath $\alpha$}}
\newcommand{\bfeta}{\mbox{\boldmath $\eta$}}
\newcommand{\bftheta}{\boldsymbol{\theta}}
\newcommand{\bfpi}{\mbox{\boldmath$\pi$}}
\newcommand{\bfnu}{\mbox{\boldmath $\nu$}}

\newcommand{\bft}{\mbox{\boldmath$t$}}


% --- Caligraphic (\cal) Commands ---

\newcommand{\calA}{{\cal A}}
\newcommand{\calE}{{\cal E}}
\newcommand{\calL}{{\cal L}}
\newcommand{\calM}{{\cal M}}
\newcommand{\calN}{{\cal N}}
\newcommand{\calF}{{\cal F}}
\newcommand{\calG}{{\cal G}}
\newcommand{\calD}{{\cal D}}
\newcommand{\calB}{{\cal B}}
\newcommand{\calH}{{\cal H}}
\newcommand{\calI}{{\cal I}}
\newcommand{\calP}{{\cal P}}
\newcommand{\calR}{{\cal R}}
\newcommand{\calQ}{{\cal Q}}
\newcommand{\calS}{{\cal S}}
\newcommand{\calT}{{\cal T}}
\newcommand{\calU}{{\cal U}}
\newcommand{\calC}{{\cal C}}
\newcommand{\calK}{{\cal K}}
\newcommand{\calX}{{\cal X}}
\newcommand{\calV}{{\cal V}}
\newcommand{\cals}{{\cal S}}

\newcommand{\ind}{\boldsymbol{1}}
\newcommand{\Exp}{\mathbb{E}}
\newcommand{\Var}{\mathbb{V}ar}


\newcommand{\eqdistr}{\stackrel{\cal D}{=}}
\newcommand{\convdistr}{\stackrel{\cal D}{\rightarrow}}
\newcommand{\convrel}{\stackrel{\cal D}{\sim}}
\newcommand{\convas}{\stackrel{\rm a.s.}{\rightarrow}}
\newcommand{\convprob}{\stackrel{\Pr}{\rightarrow}}


\begin{document}
 

\title{Introduction to simulations and Monte Carlo methods:
\\ The Theory of Generators}
\author{Pawe{\l} Lorek  \\ \textsl{Mathematical Institute, University of Wroc{\l}aw, Poland} \\
\ \\
}

%\author{AMS Special Session on Transient Probabilities of Random Processes, Duality Theory and Gambler’s Ruin Probabilities}

\date{\today}

\frame{\titlepage} 


\begin{frame}{Sequence of random numbers — formal definitions}
\textbf{On $[0,1)$}
A sequence $(U_j)_{j\ge 1}$ is a \textsl{sequence of random numbers} on $[0,1)$ if:
\begin{enumerate}
  \item[i)] Each $U_j$ has the uniform distribution $\calU[0,1)$, i.e.
  \[
    \Prob(U_j \le t) =
    \begin{cases}
      0 & t<0,\\
      t & 0\le t<1,\\
      1 & t\ge 1,
    \end{cases}
  \]
  \item[ii)] $U_1,U_2,\ldots$ are i.i.d., i.e., for $n\ge 1$,
  \[
    \Prob(U_1\le t_1,\ldots,U_n\le t_n)=t_1\cdots t_n .
  \]
  $$\bfP,\bfmu, \bfpi$$
\end{enumerate}


\smallskip
\textbf{Random bits}
$(B_j)_{j\ge 1}$ is a sequence of random bits if
$\Prob(B_j=0)=\Prob(B_j=1)=\tfrac12$ and $B_1,B_2,\ldots$ are i.i.d., so for $n\ge 1$,
\[
  \Prob(B_1=b_1,\ldots,B_n=b_n)=2^{-n}.
\]

\end{frame}

\begin{frame}{Discrete alphabet $\{0,1,\ldots,M-1\}$}
\textbf{block}{Uniform on a finite set}
A sequence $(Y_j)_{j\ge 1}$ is a sequence of random numbers on
$\{0,1,\ldots,M-1\}$ if for each $j$:
\begin{enumerate}
  \item[i)] $\Prob(Y_j=i)=\tfrac1M$ for $i=0,\ldots,M-1$,
  \item[ii)] $Y_1,Y_2,\ldots$ are i.i.d., hence for $n\ge 1$,
  \[
    \Prob(Y_1=y_1,\ldots,Y_n=y_n)=M^{-n}.
  \]
\end{enumerate}


\smallskip
\textbf{Two practical remarks}
\begin{itemize}
  \item Computers output grid values in $\{0,\tfrac1M,\ldots,\tfrac{M-1}{M}\}$ (often $M=2^k$).
  \item Any subsequence chosen by increasing indices $n_1<n_2<\cdots$ remains i.i.d.\ with the same distribution.
\end{itemize}

\end{frame}



\begin{frame}{Discrete-time Markov chain.}
Let $(X_n)_{n\ge 0}$ be a stochastic process on a countable state space $S$.
It is a \emph{Markov chain} with transition matrix $\bfP=(p_{ij})_{i,j\in S}$ if
\[
\Pr(X_{n+1}=j \mid X_n=i, X_{n-1},\ldots,X_0) \;=\; p_{ij}, \quad \sum_{j\in S} p_{ij}=1.
\]
The \emph{initial distribution} is $\bfmu=(\mu_i)_{i\in S}$ with $\mu_i=\Pr(X_0=i)$.
Then the distribution of $X_n$ is $\bfmu \bfP^n$.
\end{frame}

\begin{frame}{n-step transitions and Chapman--Kolmogorov.}
Define the $n$-step matrix $\bfP^{(n)}$ by $p^{(n)}_{ij}=\Pr(X_n=j\mid X_0=i)$.
The Chapman--Kolmogorov equations state
\[
\bfP^{(n+m)} \;=\; \bfP^{(n)} \bfP^{(m)}, \qquad \text{hence } \bfP^{(n)}=\bfP^n.
\]
For a set $A\subseteq S$, the \emph{hitting probability} from $i$ is
$h_i(A)=\Pr_i(\exists n\ge 0: X_n\in A)$.
\end{frame}

\begin{frame}{Communication, classes, irreducibility.}
We say $i\to j$ if $p^{(n)}_{ij}>0$ for some $n\ge 0$.
States $i$ and $j$ \emph{communicate} if $i\to j$ and $j\to i$ (denote $i\leftrightarrow j$).
This is an equivalence relation; its classes are \emph{communicating classes}.
A chain is \emph{irreducible} if it has a single communicating class.
\end{frame}

\begin{frame}{Periodicity and aperiodicity.}
The \emph{period} of state $i$ is
\[
d(i) \;=\; \gcd\{ n\ge 1 : p^{(n)}_{ii}>0\}.
\]
If $d(i)=1$ the state is \emph{aperiodic}. In an irreducible chain, all states share the same period.
\end{frame}

\begin{frame}{Stationary distribution and detailed balance.}
A row vector $\bfpi$ with nonnegative entries summing to $1$ is \emph{stationary} if
\[
\bfpi \bfP \;=\; \bfpi.
\]
If there exists a positive measure $\bfpi$ s.t.\ $\bfpi_i p_{ij} = \bfpi_j p_{ji}$ for all $i,j$ (detailed balance),
then $\bfpi$ (normalized) is stationary and the chain is \emph{reversible}.
\end{frame}

\begin{frame}{Convergence (ergodic theorem).}
If the chain is finite, irreducible, and aperiodic, then there is a unique stationary distribution $\bfpi$ and
\[
\lim_{n\to\infty} \bfmu \bfP^n \;=\; \bfpi \quad \text{(independent of $\bfmu$).}
\]
Moreover, for any function $f:S\to\Real$,
\[
\frac{1}{N}\sum_{n=1}^N f(X_n) \xrightarrow[N\to\infty]{\text{a.s.}} \sum_{i\in S} \bfpi_i f(i).
\]
\end{frame}

\begin{frame}{Example: two-state chain.}
Let $S=\{0,1\}$ and
\[
\bfP = \begin{pmatrix} 1-a & a \\ b & 1-b \end{pmatrix}, \qquad a,b\in(0,1).
\]
Then the unique stationary distribution is
\[
\bfpi \;=\; \left( \frac{b}{a+b}, \frac{a}{a+b} \right),
\]
and $\bfmu \bfP^n \to \bfpi$ exponentially fast as $n\to\infty$.
\end{frame}



\end{document}

